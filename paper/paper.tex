% This must be in the first 5 lines to tell arXiv to use pdfLaTeX, which is strongly recommended.
\pdfoutput=1
% In particular, the hyperref package requires pdfLaTeX in order to break URLs across lines.

\documentclass[11pt]{article}

% Remove the "review" option to generate the final version.
\usepackage[review]{acl}

% Standard package includes
\usepackage{times}
\usepackage{latexsym}

% For proper rendering and hyphenation of words containing Latin characters (including in bib files)
\usepackage[T1]{fontenc}
% For Vietnamese characters
% \usepackage[T5]{fontenc}
% See https://www.latex-project.org/help/documentation/encguide.pdf for other character sets

% This assumes your files are encoded as UTF8
\usepackage[utf8]{inputenc}

% This is not strictly necessary, and may be commented out,
% but it will improve the layout of the manuscript,
% and will typically save some space.
\usepackage{microtype}

% If the title and author information does not fit in the area allocated, uncomment the following
%
%\setlength\titlebox{<dim>}
%
% and set <dim> to something 5cm or larger.
\title{Syntax-Based Concept Alignment for Domain-Specific Machine Translation}

% Author information can be set in various styles:
% For several authors from the same institution:
% \author{Author 1 \and ... \and Author n \\
%         Address line \\ ... \\ Address line}
% if the names do not fit well on one line use
%         Author 1 \\ {\bf Author 2} \\ ... \\ {\bf Author n} \\
% For authors from different institutions:
% \author{Author 1 \\ Address line \\  ... \\ Address line
%         \And  ... \And
%         Author n \\ Address line \\ ... \\ Address line}
% To start a seperate ``row'' of authors use \AND, as in
% \author{Author 1 \\ Address line \\  ... \\ Address line
%         \AND
%         Author 2 \\ Address line \\ ... \\ Address line \And
%         Author 3 \\ Address line \\ ... \\ Address line}
% TODO: authors info
\author{First Author \\
  Affiliation / Address line 1 \\
  Affiliation / Address line 2 \\
  Affiliation / Address line 3 \\
  \texttt{email@domain} \\\And
  Second Author \\
  Affiliation / Address line 1 \\
  Affiliation / Address line 2 \\
  Affiliation / Address line 3 \\
  \texttt{email@domain} \\}

\begin{document}
\maketitle
\begin{abstract}
% TODO: abstract
\end{abstract}

\section{Introduction}
Grammar-based systems such as GF \cite{TODO:} have been successfully employed in domain-specific Machine Translation (MT). 
What makes these systems well suited to the task is the fact that, when we constrain ourselves to a specific domain, where precision is most often more important than coverage, they can provide strong guarantees in terms of grammatical correctness. 

Nevertheless, lexical exactness is, in this context, just as important as grammaticality. An important part of the design of the Controlled Natural Language (CNL) the grammar in such a system describes becomes, then, the creation of a translation lexicon. In many cases, this is done for the most part manually, resulting in a time consuming task requiring significant linguistic knowledge. When the grammar is designed based on a parallel corpus of example sentences, it is possible to automate part of this process by means of statistical word and phrase alignment techniques. None of them is, however, suitable for the common case in which only a limited number of example sentences is available.

In this paper, we propose an alternative approach to the automation of this task. While still being data-driven, our method is grammar-based and, as such, capable of extracting meaningful correspondences even from individual sentence pairs. 

A further advantage of performing the syntactic analysis of the sentences is that we do not have to choose whether to focus on the word or phrase level. Instead, we can simultaneously operate at different levels of abstraction, thus extracting both single- and multiword -even non-contiguous- correspondences. For this reason, we refer to the task our system attempts to automate as \textit{Concept Alignment} (CA). We call \textit{concepts} the abstract units of translation, composed of any number of words, the system identifies, and represent them as \textit{alignments}, i.e. pairs of semantically equivalent concrete expressions.

\section{Methodology}

\section{Evaluation}

\section{Conclusions}

% TODO: see why it does not compile
% Entries for the entire Anthology, followed by custom entries
%\bibliography{anthology,custom}
%\bibliographystyle{acl_natbib}

\end{document}
